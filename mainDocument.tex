% Kompiuterijos katedros šablonas
% Template of Department of Computer Science II
% Versija 1.0 2015 m. kovas [ March, 2015]

\documentclass[a4paper,12pt,fleqn]{article}
\input{allPacks}

\newtoggle{inLithuanian}
 %If the report is in Lithuanian, it is set to true; otherwise, change to false
\settoggle{inLithuanian}{true}

%create file preface.tex for the preface text
%if preface is needed set to true
\newtoggle{needPreface}
\settoggle{needPreface}{false}

\newtoggle{signaturesOnTitlePage}
\settoggle{signaturesOnTitlePage}{true}


\input{macros}

\begin{document}
 % #1 -report type, #2 - title, #3-7 students, #8 - supervisor
 \depttitlepage{Darbo/ataskaitos tipas}{Kibernetinio saugumo pažeidžiamos paslaugos ir atakos vektoriaus įgyvendinimo metodų tyrimas}{Deividas Slauzgalvis}{Mindaugas Strakšys} 
 {}{}{}{}% students 2-5
 {dr. Vadovas Vadovaitis}

\tableofcontents


%keywords and notations if needed
\sectionWithoutNumber{Sutartinis terminų žodynas}{keywords}{Pateikiamas terminų sąrašas (jei reikia)}

 %both abstracts
\bothabstracts{\input{abstract}}%tex-file of abstract in original language
{Darbo pavadinimas kita kalba} %if work is in LT this title should be in English
{\input{abstractEN}}%tex-file of abstract in other language


 %Introduction section: label is sec:intro
\sectionWithoutNumber{\keyWordIntroduction}{intro}
\input{introduction.tex}



 %the main part
\newpage
\section{Kibernetinio saugumo analizė}
\label{sec:motivation}
\subsection{Kibernetinio saugumo apžvalga}


\subsubsection{Kibernetinio saugumo istorija}
\subsubsection{Kompiuterinių sistemų atakavimo apžvalga}
\subsubsection{Kompiuterinių sistemų saugumo apžvalga}
\label{sec:data}
Pateikiamas trečio lygio poskyrio tekstas.
\subsection{Pagrindiniai kibernetinių atakų tipai}
\subsubsection{SQL injekcijos}
\subsubsection{Phishingas}
\subsubsection{Kenkėjiškos programos}
\subsubsection{Slaptažodžių laužimas}
\subsubsection{DDOS/DOS atakos}
\section{Pasirinktų atakos vektorių analizė}

\section{Pasirinktų atakos vektorių implementacija}

\section{Šaltiniai}
 %Conclusions section
\sectionWithoutNumber{\keyWordConclusions}{conclusion}
\input{conclusions.tex}

%ateities darbų gairės, planas/next steps of the work
\sectionWithoutNumber{Ateities tyrimų planas}{future}{Pristatomi ateities darbai ir/ar jų planas, gairės tolimesniems darbams....}

 %file literatureSources.bib
\referenceSources{literatureSources}



%% this part is optional
\newpage
\begin{appendices}
Dokumentą sudaro du priedai: \ref{app:a} priede  ....
\newpage
\section{Pirmojo priedo pavadinimas}
\label{app:a}
Pirmojo priedo tekstas ...

\newpage
\section{Antrojo priedo pavadinimas}
Antrojo priedo tekstas ...

\end{appendices}


\end{document}
