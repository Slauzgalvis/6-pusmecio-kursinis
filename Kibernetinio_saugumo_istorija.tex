Kibernetinio saugumo industrija, kuri 2015 metais buvo verta 75 milijardų dolerių, prasidėjo 1988 metais, kai Robert T.Morris paleido savo replikuojantį "kirminą" ARPANET tinkle(interneto pradininkas). Šis "kirminas" buvo dalis projekto, kuris skaičiavo interneto dydį užkrėsdamas UNIX operacines sistemas tam kad suskaičiuotų jose esančių prisijungimų prie interneto kiekį. Dėl programavimo klaidos, "kirminas" pradėjo jungtis į tas pačias mašinas daugelį kartų, taip visiškai "užkišdamas" tinklus ir priverdamas sistemas "užlūžti", taip tapdamas pirmuoju tokio tipo įrankiu susilaukusiu didžiulo žiniasklaidos dėmesio. Jo kūrėjas buvo išmestas iš universiteto, nuteistas lygtinai trejiems metams bei nubaustas 10 000 dolerių bauda. \newline Paskutiniame 20a. dešimtmetyje virusai pradėjo smarkiai plisti internetinėje erdvėjė, turbūt tada populiaurisi jų - \textbf{ILOVEYOU} ir \textbf{The Melissa} virusai. Jie užkrėtė dešimtis milijonų kompiuterių visame pasaulyje, privertė žlugti daugelį elektroninio pašto sistemų. Šie virusai pasižymėjo vienu įdomiu dalyku - jie neturėjo apibrėžto tikslo ir taikinio, ir nesiekė jokios finansinės ar politinės naudos. Vienintelis jų tikslas buvo sukelti paniką ir chaosą virtualioje erdvėje. Tam, kad pasipriešinti šiems virusams buvo pradėtos kurtis pirmosios sistemos, skirtos apsiginti nuo virusų (antivirusinės). Taipogi, kompanijos pradėjo šviesti ir edukuoti savo darbuotojus kibernetinio saugumo temomis. Jie buvo išmokyti neatidarinėti neaiškių elektroninių laiškų, tam, kad išvengti "phishing" tipo atakų. Būtent šiuo metu kompanijos rimtai susirūpino kibernetinio saugumo klausimais ir apie tai pradėta diskutuoti viešumoje. \newline 21a. pirmajame dešimtmetyje kibernetinės atakos pasidarė dar rimtesnės ir pavojingesnės. Atakuoti buvo pradėtos kreditinės kortelės ir elektroninės banko sąskaitos. Tarp 2005 ir 2007 metų Albert Gonzalez su savo kibernetinių nusikaltėlių gauja sugebėjo pavogti informaciją iš bent 45,7 milijono kreditinių kortelių. TJX kompanija dėl to patyrė 256 milijonų dolerių nuostolį. Tai buvo puiki pamoka tiek TJX, tiek kitoms kompanijoms. Į kibernetinį saugumą reikėjo pradėti žiūrėti daug rimčiau nei ankščiau, nes tai praleidus pro akis nuostoliai gali būti milžiniški ir pasibaigti kompanijos bankrotu. \newline 2014 metais kibernetiniai nusikaltėliai perėmė milžiniškus kiekius Sony bei Target kompanijų duomenų. Šįkart buvo taikytasį ne tik į finansinę naudą, bet ir į kompanijos reputaciją ir jos darbuotojų gerovę. Tačiau Target kompaniją dėl to prarado apie 162 milijonus dolerių ir daugelį klientų - nes nusikaltėliai turėjo prieigą prie 70 milijonų Target kompanijos klientų asmeninių duomenų. Taipogi nustatyta,kad už Sony kompanijos duomenų kompromitavimą atsakingą Šiaurės Korėja. Taigi, kibernetinės atakos tapo ne tik finansinės naudos siekimo įrankiu, tačiau ir stipriu politiniu(teroristiniu) įrankiu, skirtu aiškintis nesutarimus tarp šalių. Kuo technologijos modernesnės, tuo daugiau naudingos ir svarbios informacijos ten talpinama. Dabar kibernetinė sauga yra vienas svarbiausių prioritėtų kompanijose. Ir tuo rūpintis reikia, kol atakos dar neįvyko (išankstinė atakų prevencija), nes atakos eksplotavimo metu gintis jau per vėlu ir nuostaliai gali būti tragiški.
