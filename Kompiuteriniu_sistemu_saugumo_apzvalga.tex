Internetas, tai vieta, kur dažnu atvėju vartotojas pervertina savo saugumą, be išankstinio pamąstymo naršo interneto platybėse bei talpiną asmeninę informacija. Tačiau tai nėra saugi aplinka, o ypač pavojinga, nes nuo kimbernetinių atakų nė vienas kompiuterio vartotojas negali apsisaugoti visu šimtu procentu. Tačiau yra būdų, kurie padėtų atakos riziką sumažinti ar net privestų atakuotuojus nuleisti rankas.

Vienas iš svarbiausių būdų apsisaugoti nuo programuotojų, kurie rengia kibernetines atakas, yra naudojamos programinės įrangos naujinimas, kai tik nauja versijas yra išleista. Kad nepraleisti programinės įrangos naujinimų svarbu palaikyti automatinio atnaujinimo funkciją naudojamose programėlėse. Nuolatinis programinės įrangos atnaujinimas yra svarbus tuom, kad naujesnė produkto versija nebeturi senesnių versijų saugumo trukumų, kurie buvo atrasti saugumo specialistų. Taip pat užtrunka, kol programuotojai atranda saugumo spragas tik išleistoje versijoje. Taigi tai yra vienas iš svarbesnių būdų atbaidyti kenkėjus.

Kitas svarbus ir dažnas atakavimo būdas naudojant susisiekimo priemones. Šiuo atvėju svarbu apsisaugoti nuo programuotojų, kurie nori išvilioti jūsų prisijungimo duomenis komunikacijos pagalba. Tai dažniausiai yra daroma naudojant elektroninį paštą. Šiuo būdų vartotojas gauna labai oficialią žinutę, kuri iš tikrūjų yra suklastota. Žinutėje dažniausiai stengiamasi nuvilioti į programuotojų svetaines, kur programuotojas mato visus aukos vykdomus procesus. Taip pat žinutė gali talpinti užkrėstas programas ar tiesiog prašyti asmeninės informacijos. Kita susisiekimo priemonė, kurią naudoja atakų kūrėjai yra mobilus telefonas. Pasinaudojus šiuo įrenginiu atakų kūrėjai apsimeta tam tikros kompanijos atstovu ir taip megina išvilioti aukos informaciją. Šiuo atvėju svarbu išlaikyti blaivų protą, kai komunikacijos pagalba norima sužinoti svarbią asmeninę informaciją.

Taip pat populiarus būdas programuotojams sužinoti slaptos informacijos yra bevielis ryšys. Juo pagalba programuotojai viešose vietose ieško neapsaugotų elektroninių prietaisų naudojančių Wi-Fi technologiją. Antivirusinės programos dažnai apsaugo nuo šių atakų, tačiau svarbu būti atsargiems naudojantis atviru Wi-Fi signalu, o privačių naršymo procesų atlikimas, pavyzdžiui elektroninės bankininkystės naudojimasis ar kompanijos duomenų persiuntimas, tūrėtų būti paliktas tik ypač saugiems interneto tinklams(namų, įmonės internetinis ryšys).

Svarbu tai, kad atliekami procesai naudojantis internetinį ryšį ar komunikacijos platformas būtų apgalvoti ir atsakingi. Net paprasčiausias mobilaus telefono numerio nutekinimas ar viešas paskelbimas gali pridaryti žalos numerio savininkui. Todėl svarbu užtikrinti ir įsitikinti, kad slapta kompiuteryje ar internetinėje erdėje laikoma informacija yra saugi ir prieinama tik asmenims tam turintiems teisę.   