Internetas, tai vieta, kur dažnu atvėju vartotojas pervertina savo saugumą, be išankstinio pamąstymo naršo interneto platybėse bei talpiną asmeninę informacija. Tačiau tai nėra saugi aplinka, o ypač pavojinga, nes nuo kimbernetinių atakų nė vienas kompiuterio vartotojas negali apsisaugoti visu šimtu procentu. Tačiau yra būdų, kurie padėtų atakos riziką sumažinti ar net privestų atakuotuojus nuleisti rankas.

Vienas iš svarbiausių būdų apsisaugoti nuo programuotojų, kurie rengia kibernetines atakas, yra naudojamos programinės įrangos naujinimas, kai tik nauja versijas yra išleista. Kad nepraleisti programinės įrangos naujinimų svarbu palaikyti automatinio atnaujinimo funkciją naudojamose programėlėse. Nuolatinis programinės įrangos atnaujinimas yra svarbus tuom, kad naujesnė produkto versija nebeturi senesnių versijų saugumo trukumų, kurie buvo atrasti saugumo specialistų. Taip pat užtrunka, kol programuotojai atranda saugumo spragas tik išleistoje versijoje.

Taip pat svarbu
Kitas dažnai programuotojų naudojamas metodas pakenkti interneto vartotojui yra 