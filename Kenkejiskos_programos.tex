Kenkėjiškas programas galima skirstyti į tris pagrindines kategorijas:

1) Virusus

2) Kirminus

3) Trojos arklius

Nors visi jie skirti užkrėsti bei kompromituoti sistemą, kiekviena kategorija turi esminių skirtumų.

Virusas – tai kenkėjiškas kodas, prikabinamas prie įvairių failių – dažniausiai .exe (executable) tipo failų. Pagrindinis virusų trūkumas yra tai, kad jie negali aktyvuotis bei plisti be jokio vartotojo įsikišimo. Tam, kad virusas pradėtu plisti būtina sąsaja tarp vartojo ir komputerio. Tam, kad virusai plistu iš kompiuterio į kompiuterį, kompiutero vartotojas pats turi jį nusiųsti (sąmoningai arba ne) elektroniniu paštu ar kitais failų per tinklą dalinimosi metodais. Virusu darom žala gali būti įvairi, pradedant tiesiog kompiuterio sulėtinimu, baigiant programinės įrangos (o kartais ir vadinamasios „geležies“ (angl. „hardware“) bei failų sugadinimu nepataisomai. 

Kirminai iš esmės gan panašus į virusus, kartais net laikomi virusu „sub“ kategorija. Kirminai iš kompiuterio į kompiuterį gali plisti daug lengviau nei virusai – jiem nereikia žmogaus įsikišimo. Kirminai patys sugeba skleisti ir platinti save tinkle. Vienas kirminas esantis kompiuteryje gali save paplatinti tinkle šimtus tūkstančių ar net milijonus kartu – todėl kirminai daug pavojingesni negu paprasti savęs platinti tinkle negalintys virusai. Save duplikuodami kirminai sunaudja labai didelius kompiuterio bei tinklo resursus, taip priversdami kompiuterius, web serverius bei tinklų serverius lėčiau veikti ar nustoti veikti visiškai. 

Trojos arkliu vadinama kenkėjiška įranga, kuri apsimeta esanti tuo, kuo ištikrūjų nėra. Taip pavadinta dėl Trojos arklio mito. Dažniausias trojos arklys kompiuteryje apsimeta esąs naudinga programina įranga, tačiau įjungtas daro įvairią žalą kompiuteriui. Trojos arkliai dažniau būną tiesiog įkyrus negu kenksmingi,  bet būna ir išimčių – kartais jie gali ištrinti arba užkriptuoti informaciją esančia kompiuteryje. Nemaža dalis Trojos arklių taip pat gali veikti pasiremdami „backdoor“ principu, atverdami prieigą prie kompiuterio įsilaužėjams ar kitiems kenkėjiškų tikslų turintiems asmenims. Skirtingai nei virusai ar kirminai, Trojos arkliai savęs neduplikuoja kituose kompiuteriuose ar sistemose.