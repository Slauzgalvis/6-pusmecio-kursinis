Kiekviena kompiuterinė sistema turi trukumų, kuriuos anksčiau ar vėliau atranda programuotojai. Tačiau trūkumo ieškojimas nėra paprastas procesas, jis susideda iš daugybės etapų, kurie kas kart gali skirtis dėl skirtingų atakų tipų. Tačiau kiekvienos atakos pradinis taškas, nuo kurio prasideda atakos planavimas, yra pasiruošimo stadija.

Kibernetinių atakų pasiruošimo stadijoje svarbiausias žingsnis yra susipažinimas su sistema, kurią ruošiamsi atakuoti. Šiame etape svarbu susipažinti su sistemos aukos architektūra, naudojamas programas, kokių technologijų pagalba ši sistema egzistuoja internetinėje erdvėje, sistemos IP adresus ar betkokią kitą su sistema susijusią informaciją, kuri galėtų padėti lengviau ir saugiau užkrėsti sistemą. Sistemos architektūros analyzavimas gali prasidėti nuo informacijos ieškojimo aukos sistemos socialiniuose puslapiuose ar net meginant susisiekti su asmenimis atsakingais už sistemos priežiūrą ir taip išgauti reikiamų duomenų.

Sisipažinus su atakuojama sistema kitas žingsnis yra sistemos skanavimas naudojant konkrečius tinklo ir jame esančių paketų analizavimo įrankius. Tokiems įrankiams dažnu atvėju reikalingas aukos sistemos IP adresas, su kuriuo pagalba analizavimo įrankis sužinotu plačiau apie techninę sistemos veikimo pusę. Išgautoje informacijoje galima aptikti naudojamą operacinę sistemą, naudojamų programinių įrankių versijas, maršrutizatoriaus lenteles(angliškai. routing tables). Visa analizavimo įrankio surinkta informacija leidžia susidėlioti konkretesnį atakuojamos sistemos architektūros braižą ir suteikia galimybę lengviau atrasti sistemos saugumo spragų.

Pasinaudojus išgauta informacija vienas iš dažniausių būdų užpulti sistemą yra pasinaudojimas atakuojamos sistemos naudojamų programinių įrankių saugumo spragomis. Pagal naudojamą programinės įrangos versija internete galima surasti jau atrastas saugumo spragas ir jomis pasinaudoti. Vienos saugumo spragos padeda padaryti nedaug žalos, tačiau su kitomis galima išgauti privačios informacijos.
