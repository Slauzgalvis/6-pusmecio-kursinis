Kiekviena kompiuterinė sistema turi trukumų, kuriuos anksčiau ar vėliau atranda programuotojai. Tačiau trūkumo ieškojimas nėra paprastas procesas, jis susideda iš daugybės etapų, kurie kas kart gali skirtis dėl skirtingų atakų tipų. Tačiau kiekvienos atakos pradinis taškas, nuo kurio prasideda atakos planavimas, yra pasiruošimo stadija.

Kibernetinių atakų pasiruošimo stadijoje svarbiausias žingsnis yra susipažinimas su sistema, kurią ruošiamsi atakuoti. Šiame etape svarbu sužinoti sistemos aukos architektūra, naudojamas programas bei kokių technologijų pagalba ši sistema egzistuoja internetinėje erdvėje.